\documentclass[10pt,a4paper]{article}
\usepackage[utf8]{inputenc}
\usepackage{amsmath}
\usepackage{amsfonts}
\usepackage{amssymb}
\usepackage{graphicx}
\author{Pranav Satheesh}
\title{Solving ordinary differential equations numerically}

\begin{document}
\maketitle


\section{First order ODE}

A general initial value problem of first order ODE are of the form 

\begin{equation}
    \frac{dy}{dx} = f(x,y)
    \label{eq:ODE}
\end{equation} with the inital value at a point specified $y(x_0) = y_0$.




There are two types of methods used to get a solution of first-order equations:

\begin{itemize}
    \item  A series for y in terms of powers of x, from which the value of y
    can be obtained by direct substitution.
    \item   A set of tabulated values of x and y.   
\end{itemize}

i) - Taylor and Picard
ii) Euler, Runge-Kutta, Adams-Bashforth  - called step-by-step methods


Problems in which all the initial conditions are specified at the initial point only are called
initial value problems. A differential equation of the nth order will requre n initial conditions.

First order - one conditions
Problems involving second-and higher-order differential equations, we may prescribe the conditions at two
or more points. Such problems are called boundary value problems.


\section{Taylor's Series}

For the general first-order ODE $y^{'} = f(x,y)$ with the initial condition 
$y(x_0) = y_0$ we can give a solution by taylor expanding $y(x)$ around $x = x_0$.

\begin{equation}
    y(x) = y_0 + (x - x_0) y^{'}_{0} + \frac{(x-x_{0})^2}{2!} y^{''}_{0} + \cdots
    \label{eq:Taylor}
\end{equation}

We now need to evalue the higher orders $y^{'}_{0},y^{''}_{0},\cdots$. 

\begin{align}
    y^{''} &= f_{x} + f f_{y} \\
    y^{''} &= f_{xx} + 2 f f_{xy} + f^2 f_{yy} + f_{x} f_{y} + f f^{2}_{y}\\
\end{align}




\section{Euler's Method}

The methods previously discussed give power series solution. But if we require
to have a table of $(x,y)$ values by solving the ODE numericaly, we need to use step-by-step methods

We will solve by takinf steps of the form $x = x_{r} + r h$ where
$r = 1,2,\cdots$. Integrating  we get


\begin{equation}
    y_{1} = y_{0} + \int_{x_{0}}^{x_{1}} f(x,y) dx
    \label{eq: integral}
\end{equation}

We will assume as an approximation that $f(x,y) = f(x_{0},y_{0})$ in the interval
$x_{0} \leq x \leq x_{1}$.

This gives us 

\begin{equation}
    y_{1} \approxeq y_{0} + h f(x_{0},y_{0})
\end{equation}

Continuing the same way, we get the general formula for Euler's method


\begin{equation}
    y_{n+1} = y_{n} + h f(x_{n},y_{n}), \, n = 0,1,2,\cdots
\end{equation}

\subsection{Error estimates for Euler Method}




\subsection{Modified Euler method}

In the Euler's method we approximated $f(x,y)$ in the interval and now we 
approximate the integral \eqref{eq: integral} by trapezoidal rule 

\begin{equation}
    y_{1} = y_{0} + \frac{h}{2} [ f(x_{0},y_{0}) + f(x_{1},y_{1})]
\end{equation}

We get the general iteration formula

\begin{equation}
    y_{1}^{(n+1)} = y_{0} + \frac{h}{2} [ f(x_{0},y_{0}) + f(x_{1},y_{1}^{(n)})], \, n = 0,1,2.\cdots
\end{equation}

whwre $y_{1}^{(n)}$ is the nth approximation to $y_{1}$ and $y_{1}^{0}$
is chosen from Euler's formula:

\begin{equation}
    y^{(0)}_{1} = y_{0} + h f(x_{0},y_{0})
\end{equation}



\end{document}
